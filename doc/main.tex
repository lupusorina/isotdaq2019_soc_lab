
\documentclass{article}

\usepackage[colorlinks=true,urlcolor=blue,linkcolor=blue]{hyperref}
\usepackage{cleveref}
\usepackage{geometry}
\geometry{a4paper, margin=0.6in}
\usepackage{minted}
\usepackage{graphicx}
\newcommand{\HRule}{\rule{\linewidth}{0.5mm}}
\usepackage{wrapfig}
\usepackage{subcaption}
\usepackage{setspace}
\usepackage{booktabs}
\usepackage[T1]{fontenc}
\usepackage[font=small, labelfont=bf]{caption}
\usepackage[protrusion=true, expansion=true]{microtype}
\usepackage[english]{babel}
\usepackage{sectsty}
\usepackage{url, lipsum}
\usepackage{tcolorbox}
\usepackage{lipsum}
\usepackage[toc,page]{appendix}
\usepackage{charter}
\setlength{\parindent}{1em}
\renewcommand{\baselinestretch}{1.2}

\crefname{appsec}{Appendix}{Appendices}

\usepackage[framemethod=TikZ]{mdframed}
\usepackage{lipsum}
\mdfdefinestyle{MyFrame}{%
    linecolor=blue,
    outerlinewidth=2pt,
    roundcorner=20pt,
    innertopmargin=\baselineskip,
    innerbottommargin=\baselineskip,
    innerrightmargin=20pt,
    innerleftmargin=20pt,
    backgroundcolor=red!20!white}


\newenvironment{myexampleblock}[1]{%
    \tcolorbox[beamer,%
    noparskip,breakable,
    colback=LightGreen,colframe=DarkGreen,%
    colbacklower=LimeGreen!75!LightGreen,%
    title=#1]}%
    {\endtcolorbox}

\newenvironment{myalertblock}[1]{%
    \tcolorbox[beamer,%
    noparskip,breakable,
    colback=LightCoral,colframe=DarkRed,%
    colbacklower=Tomato!75!LightCoral,%
    title=#1]}%
    {\endtcolorbox}

\newenvironment{myblock}[1]{%
    \tcolorbox[beamer,%
    noparskip,breakable,
    colback=white,colframe=DarkBlue,%
    %colbacklower=DarkBlue!75!LightBlue,%
    title=#1]}%
    {\endtcolorbox}


\newenvironment{myitemize}
{ \begin{itemize}
    \setlength{\itemsep}{0pt}
    \setlength{\parskip}{0pt}
    \setlength{\parsep}{0pt}     }
{ \end{itemize}                  }









\begin{document}

\title{ \LARGE \normalsize {The International School of Trigger and Data Acquisition 2018} \vspace{8cm} \\
		\LARGE \textbf{\uppercase{System on Chip Laboratory}} \\ 
		\normalsize \today \vspace*{2\baselineskip}}

\date{}

\maketitle

\begin{figure}[h!]
    \centering
    \includegraphics[width=0.7\textwidth]{img/zturn.png}
\end{figure}


\newpage
\tableofcontents
\newpage

%-------------------------------------------------------------------------------
% Section title formatting
\sectionfont{\scshape}
%-------------------------------------------------------------------------------

%-------------------------------------------------------------------------------
% BODY
%-------------------------------------------------------------------------------

\section{OVERVIEW}



The System on Chip(SoC) is getting more and more popular, exceeding the capabilities of a simple microcontroller. SOCs are used mostly in smartphones and tablets due to their low power consumption, much shorter wiring and high level of integration.

This laboratory will thus aim to familiarize you with the \textbf{All Programmable System on Chip (AP SoC)}. The board (presented in Figure \ref{fig:zturnboard}) that you are going to use is called Z-TURN, built around the Xilinx Zynq-7010. It contains an FPGA and a dual-core ARM microcontroller, basically it is a \textbf{System on Chip}.




\begin{figure}[h!]
    \centering
    \includegraphics[width=0.9\textwidth]{img/zturn_details.jpg}
    \caption{Z-Turn board capabilities}
    \label{fig:zturnboard}
\end{figure}






After fulfilling this lab, you are mainly going to learn:
\begin{itemize}
\item the workflow of designing an application on FPGA
\item the interaction between an FPGA and an ARM microcontroller
\item getting an overview of the challenges of using this system
\end{itemize}




\newpage

\section{Applications}


In this laboratory, you will implement the workflow for designing one application used in
High Energy Physics.


ADD MANU's WORK


{\color{red} Explain to the students what is PWM and how does it work to control an LED.}

\subsection{Pulse Width Modulation (PWM) Slave Peripheral for LED control}


In Figure \ref{fig:pwm_waveform}, you can see a basic PWM waveform.  

The FPGA runs at a clock frequency (for this application, we will choose 50 MHz). We can create a PWM waveform  from this clock frequency, by counting a number of clock cycles. There are two important parameters that characterize the PWM waveform: \textbf{the duty cycle} and the \textbf{PWM period}. The PWM  period is set by "waiting/counting" a number of clock cycles,  while the duty cycle  tells us, in one period, how many clock cycles the PWM signal is low.


\begin{figure}[h!]
    \centering
    \includegraphics[width=\textwidth]{img/pwm_waveform.png}
    \caption{PWM waveform (Source: Application Note 333, Altera)}
    \label{fig:pwm_waveform}
\end{figure}

Now let's see how we are going to use these two simple parameters to make an RGB LED change its color.

\begin{itemize}
\item \textbf{RGB LED}

In the case of RGB LED, we need 3 PWMs signals (one per each color), such that the brightness of each of the three LEDs can be controlled independently.

\textbf{\textit{How do we choose the PWM period?}} The driving frequency of the PWM should be fast enough to avoid the flicker effect. A normal human being sees this effect until up to 100 to 150 Hz, so a higher frequency should be better to avoid this effect. For this exercise, we will use a frequency of 1.5 kHz. Let's make some calculations.

The main clock frequency of the FPGA is 50 MHz. We want to use a 1.5 kHz frequency. How much do we need to count?

\begin{equation}
counter = \frac{50000000}{15000} = 30000
\end{equation}

\textbf{\textit{How do we choose the PWM duty cycle?}} The brightness of the LEDs is controlled by the duty cycle. This is up to you to do experiments. You can choose basically any duty cycle in the range of 0$\%$ to 100$\%$.



%\noindent\rule{18cm}{1pt}

\clearpage



\subsubsection{Implementation}

In order to start our implementation, the next steps should be followed:

\begin{enumerate}
\item Open an Ubuntu terminal and type 
\begin{tcolorbox}
\begin{minted}{c}
vivado &
\end{minted}
\end{tcolorbox}

Vivado Design Suite is launched and a Getting Started Page is displayed.

\item Create a new project and set the project name to \textbf{detector}. 

Select \textbf{Next} and tick \textbf{RTL Project}. 
Next, you will be asked to add sources to your design. For the moment, we will keep the project empty. However, we need to specify \textbf{VERILOG} for both \textit{Target Language} and \textit{Simulator Language}. Press \textbf{Next}, and you will be asked about Existing IP and Constrains files. Keep them both empty again.

Now, we need to choose the platform where we implement our design. The development board is entitled MYS-7Z010-C-S Z-turn (Figure \ref{fig:board}).  Press \textbf{Next} and \textbf{Finish}.  Now, the project will initialize.
\begin{figure}[h!]
    \centering
    \includegraphics[width=0.85\textwidth]{img/board.png}
    \caption{Board Selection}
    \label{fig:board}
\end{figure}
 
\item  The \textbf{detector} project should now look  similar to Figure \ref{fig:first_page}. In this project, we will implement the necessary files to control an RGB LED.

\begin{figure}[h!]
    \centering
    \includegraphics[width=0.85\textwidth]{img/project_first_page.png}
    \caption{Project Structure}
    \label{fig:first_page}
\end{figure}


\item Create the Block Design. On the left, in the Flow Navigator window, choose \textbf{Create Block Design} and call it \textbf{detector_bd}. Press OK, and now the environment should look like in Figure \ref{fig:diagram_second_page}. 

{\color{red} Explain to the students that Vivado is based on an IP-centric design flow and why it is like this. Explain the AXI protocol used in System on Chips}


\begin{figure}[h!]
    \centering
    \includegraphics[width=0.85\textwidth]{img/project_first_page.png}
    \caption{Block Diagram}
    \label{fig:diagram_second_page}
\end{figure}



\item Now we will add the AXI Master (Zynq7 Processing System) and the PWM IP () in the block diagram. The PWM IP was already implemented for you, and you need to add it to the library. Click on the \textbf{IP settings} icon like in Figure X. Choose \textbf{Repository Manager} from the tabs. Click on $+$ symbol and go into the following path: /ip/pwm_verilog_1.0. Click Select and then OK.

Now the PWM folder is added to your project so you can choose it from the library, as in the pictures below. Press Add IP \includegraphics[width = 0.5cm]{img/icon_add.png}. 
 
 \begin{figure}[h!]
\centering
\begin{minipage}{.425\textwidth}
  \centering
  \includegraphics[width=0.8\linewidth]{img/diag_processor.png}
\end{minipage}%
\begin{minipage}{.425\textwidth}
  \centering
  \includegraphics[width=0.8\linewidth]{img/diagram_pwm.png}
\end{minipage}
\caption{IP Blocks for Master and Slave}
\label{fig:diagram}
\end{figure}


Of course, there are other important IPs that should be added, such as AXI Interface or Processor System Reset, as well as the interconnectivity between them. However, the advantage of using Block Design is that these extra work is automated for you. 


Therefore, you should press {\color{blue}\underline{Run Block Automation}} and {\color{blue} \underline{Run Connection Automation}}, let the settings as default, press OK and let the Designer Assistance to do this work on your behalf. The window should now look similar to Figure \ref{fig:block_diagram_system}. However, the Design Assistant is placing the blocks in a not-very-organized way. For a better view, we advise you to click on the Regenerate Design button \includegraphics[width = 0.5cm]{img/icon_refresh.png} in the right Menu. Looks better, right?

\begin{figure}[h!]
    \centering
    \includegraphics[width=0.85\textwidth]{img/block_automation_1.png}
    \caption{Block Diagram of our System}
    \label{fig:block_diagram_system}
\end{figure}


\item Let's dig into how the PWM Peripheral was designed in VHDL.

Right click on the PWM\_0 IP and select \textit{Edit in IP Packager}. Another project (Figure \ref{fig:ip}) will open that contains the files generated for this IP. Vivado Design Suite provides you with \textit{Packaging Steps} to configure the current IP. In the Source part, you will see an hierarchical design in VHDL. The top level is called \textit{PWM\_v1\_0.vhd} and contains an instantiation of the  \textit{PWM\_v1\_0\_S00\_AXI.vhd} where the logic is implemented. If you open  \textit{PWM\_v1\_0\_S00\_AXI.vhd}, you will see the registers specific to the AXI-Memory Map architecture (Figure \ref{fig:axi_mm_stream}). In addition, the user can add its own logic and signals. Thus, you can see the implementation of PWM outlined by the following comments:

{\color{red} Can you explain the code a bit to the students?}

\begin{minted}{vhdl}
 -- User to add parameters/ports/logic here
 ...
 -- User parameters/ports/logic ends
\end{minted}


\begin{figure}[h!]
    \centering
    \includegraphics[width=0.85\textwidth]{img/ip.png}
    \caption{IP Project Structure}
    \label{fig:ip}
\end{figure}

Now, let's close the project and return to our beautiful Block Design.



 \item Create PWM output ports
 
We need to create PWM outputs, in order to connect them to our board. Click on each PWM output and press CTRL-K or Create Port \includegraphics[width = 0.5cm]{img/make_exernal.png} symbol from the Menu. Do so for all the 3 ports. Now your IP should look like in Figure \ref{fig:pwm_logic_block}. 



 \item Let's configure the ZYNQ7 Processing System
 
Double click on the ZYNQ7 Processing System IP will open a user-friendly interface for selecting the clocks/memory/IOs. Now, you need to follow the next sub-steps:
\begin{itemize}

\item  \textbf{Zynq Block Diagram} 

Take some time to look at the diagram in order to understand the interface between PS and PL. Ask your tutor for further details, as this diagram is extremely important for a better understanding of the system.

Go to \textbf{Peripherals I/O Pins} tab.

\item \textbf{Peripherals I/O Pins}
1. Select UART, as we want to transmit through serial interface(USB-UART) different messages from the board. 

2. Select I2C1 (which will be used to talk to our light detector)


\noindent\rule{16.5cm}{1pt}

\noindent\begin{minipage}{.1\textwidth}
  \centering
  \includegraphics[height=1.5cm]{img/icon.png}
\end{minipage}
\begin{minipage}{.8\textwidth}
Verify which UART (0 or 1) can be used in our case using the USB-UART cable. Look into the datasheet of the board: \url{http://www.myirtech.com/download/Zynq7000/Z-TURN_SCH_V4_20150326.pdf}
\end{minipage}%

\noindent\rule{16.5cm}{1pt}

\item \textbf{MIO Configuration}

If you did everything correctly in the previous step, now you should be able to see only UART selected.


\item \textbf{Clock Configuration}
Let everything as default, apart from \textbf{PL Fabric Clocks}. Roll the clocks down and deselect FCLK\_CLK1. As well, change the FCLK\_CLK0 to 50MHz.


\item \textbf{DDR Configuration}
Delect \textbf{Enable DDR}
\end{itemize}

Now press OK. Your ZYNQ7 Processing System should look more simplistic now, and easier to understand. 



Therefore, you should press {\color{blue}\underline{Run Block Automation}} and {\color{blue} \underline{Run Connection Automation}}, let the settings as default, press OK and let the Designer Assistance to do this work on your behalf. The window should now look similar to Figure \ref{fig:block_diagram_system}. However, the Design Assistant is placing the blocks in a not-very-organized way. For a better view, we advise you to click on the Regenerate Design button \includegraphics[width = 0.5cm]{img/icon_refresh.png} in the right Menu. Looks better, right?


\item Create VHDL Wrapper

From the block design, we need to create the VHDL code. Therefore, in the Project Sources - Design Sources, you need to right click on the \textbf{Peripherals Peripherals.bd} and select \textit {Create HDL Wrapper}. Let the default option and press OK. Now the VHDL Wrapper should have been created. Verify that the signals PWM\_RED, PWM\_BLUE, PWM\_GREEN and PWM\_BUZZER appear instantiated. 

 \item Create Constraint File
 
The Constraint File is used in the implementation phase. After the synthesis will be run, and the available top-level nets are thus "known", they will be thus matched to the "real" pins.
  
Right click on the Constraints directory and Add Source (Add or Create Constraints), then click finish, after you give the file a name.

Open it and write the following code that will define the nets and their dedicated ports.

\begin{minted}{vhdl}
## LED RED
set_property IOSTANDARD LVCMOS33 [get_ports PWM_RED]
set_property PACKAGE_PIN XX [get_ports PWM_RED]

## LED BLUE
set_property IOSTANDARD LVCMOS33 [get_ports PWM_BLUE]
set_property PACKAGE_PIN XX [get_ports PWM_BLUE]

## LED GREEN
set_property IOSTANDARD LVCMOS33 [get_ports PWM_GREEN]
set_property PACKAGE_PIN XX [get_ports PWM_GREEN]

## LED BUZZER
set_property IOSTANDARD LVCMOS33 [get_ports PWM_BUZZER]
set_property PACKAGE_PIN XX [get_ports PWM_BUZZER]

set_property CFGBVS VCCO [current_design]
set_property CONFIG_VOLTAGE 3.3 [current_design]

\end{minted}

\noindent\rule{16.5cm}{1pt}

\noindent\begin{minipage}{.1\textwidth}
  \centering
  \includegraphics[height=1.5cm]{img/icon.png}
\end{minipage}
\begin{minipage}{.8\textwidth}
Open the next design file of the board and check Table 1-2 (page 5). Match the Default Function with our application and replace the XX with the specific pin (BGA column)
\url{http://www.myirtech.com/download/Zynq7000/Z-turnBoard.pdf}
\end{minipage}%

\noindent\rule{16.5cm}{1pt}


\item Generate the bitstream and export the hardware


Now, the project needs to be synthesized, implemented and finally the bitstream must 
be generated.
The bitstream is, as the name says, a binary sequence which indicates which connections inside the FPGA must be activated. That's why it is politically corect to call "configure an FPGA" and not "programme an FPGA".

To generate the bitstream, look into the Flow Manager on the left of the screen and press \textbf{Generate Bitstream (Figure \ref{fig:export_hardware_generate_bitstream})}. You will need to wait a bit longer until the bitstream is created. Check for errors in the process and solve them. Now, that the bitstream was successfully created, we need to export the hardware such as we can interact with it using the ARM dual-core microcontroller. 

To export the hardware, press \textbf{File $>$ Edit $>$ Export $>$ Export Hardware}. Tick \textit{Include bitstream} like in Figure \ref{fig:export_hardware_generate_bitstream}.


 \begin{figure}[h!]
\centering
\begin{minipage}{.425\textwidth}
  \centering
  \includegraphics[width=0.8\linewidth]{img/export_hardware.png}
\end{minipage}%
\begin{minipage}{.425\textwidth}
  \centering
  \includegraphics[width=0.8\linewidth]{img/generate_bitstream.png}
\end{minipage}
\caption{a) Export Hardware Window b) Generate Bitstream in Flow Manager}
\label{fig:export_hardware_generate_bitstream}
\end{figure}


\newpage
 \item Let's write some C code 

From now on, we will create the link between the fabric and the dual-core ARM microcontroller. For this, we will use \textbf{Xilinx Software Development Kit (XSDK)} which is the Integrated Design Environment for creating embedded applications. If you worked in Eclipse before, XSDK shall be a piece-of-cake, as it is build on the standard Eclipse IDE, with plug-ins and Xilinx-dedicated tools. 

To open it, press \textbf{File $>$ Launch SDK}. Now, the hardware shall be automatically imported in XSDK and now you can create a software application (Figure \ref{fig:sdk_first_page}).



 
 \begin{figure}[h!]
    \centering
    \includegraphics[width=0.85\textwidth]{img/xsdk_first_page.png}
    \caption{XSDK First Page after bitstream was exported}
    \label{fig:sdk_first_page}
\end{figure}



 \item Our first "Hello World Application"

Select \textbf{File $>$ New $>$ Application Project}.

The New Project dialog box opens. As Project Name, type the name desired, for example
\textit{LED\_Buzzer\_Controller} (Figure \ref{fig:start_proj_xsdk}). Press \textbf{Next}. From the Available Templates, choose \textit{"Hello World"}. 

Click \textbf{Finish}. 


 \begin{figure}[h!]
\centering
\begin{minipage}{.425\textwidth}
  \centering
  \includegraphics[width=0.8\linewidth]{img/app_project.png}
\end{minipage}%
\begin{minipage}{.425\textwidth}
  \centering
  \includegraphics[width=0.8\linewidth]{img/hello_world.png}
\end{minipage}
\caption{Starting a project in XSDK}
\label{fig:start_proj_xsdk}
\end{figure}




\item Connect the board

To run the first application, you need to do the following simple steps on the hardware site:
\begin{myitemize}
\item Ensure that your board is powered on by connecting the Mini-USB to USB cable from computer to the USB\_UART mini-usb socket. This connectivity will also allow us transmit and receive data through serial communication.
\item Connect the programmer on the JTAG port. Make sure the connector matches the pinout from Figure \ref{fig:board_pinout}.

 
 \begin{figure}[h!]
    \centering
    \includegraphics[width=0.4\textwidth]{img/boardpinout.png}
    \caption{JTAG Pinout}
    \label{fig:board_pinout}
\end{figure}


\end{myitemize}

\item Run the first application

Open the project source files (in the \textbf{src} folder), and double-click on the helloworld.c. 
In that file, add the next sequence before the printing line of code. In this way, you will see the output running all the time on the terminal.

\begin{minted}{c}
while(1)
\end{minted}

Find the following icon to build the project \includegraphics[width = 0.6cm]{img/run/build.png} and click on the downward arrow. Make sure the Release version is selected and then build it. Verify that the build is successful.



Next, we need to setup the programming chain, as firstly the FPGA is programmed (bitstream loaded) and then the application is loaded (.elf file).



Go into \textbf{Run > Run configurations} and do the settings from Figures \ref{fig:run_part1} and \ref{fig:run_part2} below:

 \begin{figure}[h!]
    \centering
    \includegraphics[width=0.7\textwidth]{img/run/run_part1.png}
    \caption{Target Setup}
    \label{fig:run_part1}
\end{figure}

 \begin{figure}[h!]
    \centering
    \includegraphics[width=0.7\textwidth]{img/run/run_part2.png}
    \caption{Application Setup}
    \label{fig:run_part2}
\end{figure}

Wait until the system is configured and programmed.

\item See the output

Open a terminal \textbf{CTRL+ALT+T} and open the picocom terminal as following: 

\begin{tcolorbox}
\begin{minted}{c}
picocom -b 115200 /dev/ttyUSB0
\end{minted}
\end{tcolorbox}

Make sure you put the correct serial device "ttyUSBX", according to your system.

Now, you shall see a long list of "Hello World!"

\item Let's run a more complex application.

At the beginning of this laboratory, we presented two applications we are going to implement. For simplicity, we provided you with the source code for this application that can be seen in the Laboratory folder or in Appendix \ref{appendix:C}. You just need to run it and understand the code. 


The source code contains two $*.c$ files and two header files.


$itoa_fcn.c$ - contains utility function used to convert a number to a string to be sent to UART using "print" function

$main.c$ - contains the main logic of the programme


To run it, firstly, you need to delete the \textit{helloworld.c}. Then, select the \textit{src} folder, and import the aforementioned code by right click and selecting \textbf{Import > General > File System}. Add the Laboratory folder like in Figure \ref{fig:import}.

 \begin{figure}[h!]
    \centering
    \includegraphics[width=0.7\textwidth]{img/run/import.png}
    \caption{Import source files}
    \label{fig:import}
\end{figure}

Now, run the application similarly and HAVE FUN! (before moving to the LINUX part)

\end{enumerate}



\clearpage

\section{LINUX OS on System on Chip}


\subsection{First Interaction with LINUX on System on Chip}

Now, we will move on with running an operating system on the System on Chip. For this, we will do the following steps:
\begin{myitemize}
\item you disconnected the JTAG cable used previously to programme the board
\item the mini SD Card is inserted in its socket
\item find Jumper 1 and Jumper 2 (JP1 and JP2) and make sure they are in the following order:
\begin{minted}{c}
JP1 OFF 
JP2 ON
\end{minted}
\end{myitemize}


When previously steps are done, you will see the RGB LED blinking on the board. It is very
important to make these configurations before moving on to the next section of the lab.

\subsection{Further configurations}

The SD card contains the image of Ubuntu 12.04 and will boot on the board at every 
restart.

\subsubsection{Screen}

\subsubsection{File transfer between host computer and the board}

% Connect a screen to see that it works

% Open a terminal and communicate 

% Send a file through SSH

\subsection{Run the first application}

In this part, we will do all the procedures to blink the RGB LED that is available 
on the board, as well as to read the accelerometer values.

\subsection{Webserver}
Now, we will learn how to display the accelerometer values on an webserver running
on the host computer.







\clearpage

\begin{appendices}
\crefalias{section}{appsec}
\section{Setting up the picocom terminal}
\label{appendix:graph}
\begin{myitemize}
    \item Install Picocom:

    \begin{tcolorbox}
        \begin{minted}{c}
        sudo apt-get install picocom
        \end{minted}
    \end{tcolorbox}

    \item Find the device dev path for the board:

    \begin{tcolorbox}
        \begin{minted}{c}
        dmesg | grep tty
        \end{minted}
    \end{tcolorbox}

    \item Give permission to users to write and read, presuming that the device dev path is ttyUSB0

    \begin{tcolorbox}
        \begin{minted}{c}
        sudo chmod 666 /dev/ttyUSB0
        \end{minted}
    \end{tcolorbox}

    \item Start picocom with a baudrate of 115200 

    \begin{tcolorbox}
        \begin{minted}{c}
        picocom -b 115200 /dev/ttyUSB0
        \end{minted}
    \end{tcolorbox}

    \item To exit Picocom: CTRL-A and CTRL-X


\end{myitemize}


\clearpage
\section{Connecting the board to a network}
\label{appendix:b}
\begin{myitemize}
\item{Get the board connected to Internet} 

To be able to get the board connected to the Internet, you need to use the laptop as a NAT box. This is done by enabling “Share internet connection via ethernet port” on the laptop. The (only) ethernet interface of the laptop is connected to the board. The connection to the internet is done through the wireless (which should not have 802.1x enabled).



On the side of the board, we configure the network adaptor in the file /etc/network/ interfaces as following:
    \begin{tcolorbox}
    \begin{minted}{c}
    auto eth0
    iface eth0 inet static
    address 192.168.2.2
    netmask 255.255.255.0 gateway 192.168.2.1 dns-nameservers 192.168.2.1
    \end{minted}
    \end{tcolorbox}

After adding these lines to the file, the network stack must be started: 

\begin{tcolorbox}
    \begin{minted}{c}
    service networking start
    \end{minted}
\end{tcolorbox}

(this will also be done automatically at boot time).


\item{Test the network by running:}


\begin{tcolorbox}
    \begin{minted}{c}
    apt-get update
    ping 8.8.8.8
    \end{minted}
\end{tcolorbox}


\item{Getting ssh to work}

A bit more conveniënt way to have a terminal connection is directly over the network with ssh. This makes the installation of ssh necessary on the board:


\begin{tcolorbox}
    \begin{minted}{c}
    apt-get install ssh
    \end{minted}
\end{tcolorbox}

Next , set the root password with the "passwd" command.
Now, from the NAT host (i.c. your laptop), it is possible to connect though ssh:


\begin{tcolorbox}
    \begin{minted}{c}
    ssh -l root 192.168.2.2 
    \end{minted}
\end{tcolorbox}

The output will be : 


    \begin{minted}{c}
    root@localhost:~#
    \end{minted}

\end{myitemize}







\clearpage
\section{Cheating Sheet - Final Code for FPGA PART}
\label{appendix:C}
\textbf{main.c}
\begin{minted}{c}
#include "xparameters.h"
#include "xil_io.h"
#include "stdlib.h"
#include "itoa_fcn.h"

#define MY_PWM_MEMORY_MAP 0x43C00000 //This value is found in the Address editor tab in Vivado (next to Diagram tab)
#define MY_PWM_MEMORY_MAP_OFFSET 4
#define FREQUENCY_FPGA 50000000 // 50 MHz

void print_memory_mapped_registers();

// Generate a random number between 1 and 3 (1 - red; 2 - green;  3 - blue)
// Every time the red color is showed, activate the buzzer for 1 seconds
int main(){

    int red_pwm = 0;
    int blue_pwm = 0;
    int green_pwm = 0;
    int buzzer_pwm = 0;


    int led_duty_cycle_max = 12000;
    int buzzer_pwm_value =  led_duty_cycle_max;

    int rand_choose_color;
    int i;
    while(1){
        rand_choose_color = (rand() % 3) + 1;
        print("Generated random color: ");
        switch (rand_choose_color){
                case 1:
                    red_pwm = 0;
                    blue_pwm = led_duty_cycle_max;
                    green_pwm = led_duty_cycle_max;
                    buzzer_pwm = buzzer_pwm_value;
                    print("RED");
                    print("\n");
                break;

                case 2:
                    red_pwm = led_duty_cycle_max;
                    blue_pwm = 0;
                    green_pwm = led_duty_cycle_max;
                    buzzer_pwm = 0;

                    print("BLUE");
                    print("\n");
                break;

                case 3:
                    red_pwm = led_duty_cycle_max;
                    blue_pwm = led_duty_cycle_max;
                    green_pwm = 0;
                    print("GREEN");
                    print("\n");
                default:
                    red_pwm = led_duty_cycle_max;
                    blue_pwm = led_duty_cycle_max;
                    green_pwm = 0;
                    buzzer_pwm = 0;

        }


        Xil_Out32(MY_PWM_MEMORY_MAP, red_pwm);
        Xil_Out32((MY_PWM_MEMORY_MAP + MY_PWM_MEMORY_MAP_OFFSET), blue_pwm);
        Xil_Out32((MY_PWM_MEMORY_MAP + 2 * MY_PWM_MEMORY_MAP_OFFSET), green_pwm);
        Xil_Out32((MY_PWM_MEMORY_MAP + 3 * MY_PWM_MEMORY_MAP_OFFSET), buzzer_pwm);

        print_memory_mapped_registers();

        for(i=0;i < 3 * FREQUENCY_FPGA; i++){
            if (i == FREQUENCY_FPGA)
                Xil_Out32((MY_PWM_MEMORY_MAP + 3 * MY_PWM_MEMORY_MAP_OFFSET), 0);
        }
    }
}

void print_memory_mapped_registers(){
    int register_red_pwm;
    int register_green_pwm;
    int register_blue_pwm;
    char buffer_red[10];
    char buffer_blue[10];
    char buffer_green[10];

    register_red_pwm = Xil_In32(MY_PWM_MEMORY_MAP);
    register_green_pwm = Xil_In32(MY_PWM_MEMORY_MAP);
    register_blue_pwm = Xil_In32(MY_PWM_MEMORY_MAP);

    itoa_fcn(register_red_pwm, buffer_red);
    itoa_fcn(register_green_pwm, buffer_green);
    itoa_fcn(register_blue_pwm, buffer_blue);

    print("(REG READ at address 0x43C00000): Duty Cycle for RED: = ");
    print(buffer_red);
    print("\n");
    print("(REG READ at address 0x43C00004): Duty Cycle for GREEN: = ");
    print(buffer_green);
    print("\n");
    print("(REG READ at address 0x43C00008): Duty Cycle for BLUE: = ");
    print(buffer_blue);
    print("\n");
    print("---------------------------------------------------------");

}

\end{minted}



\textbf{itoa.c}
\begin{minted}{c}

#include "itoa_fcn.h"
#include "string.h"
void reverse(char s[]);
void itoa_fcn(int n, char s[])
 {
     int i, sign;

     if ((sign = n) < 0)  /* record sign */
         n = -n;          /* make n positive */
     i = 0;
     do {       /* generate digits in reverse order */
         s[i++] = n % 10 + '0';   /* get next digit */
     } while ((n /= 10) > 0);     /* delete it */
     if (sign < 0)
         s[i++] = '-';
     s[i] = '\0';
     reverse(s);
}

void reverse(char s[])
{
    int i, j;
    char c;

    for (i = 0, j = strlen(s)-1; i<j; i++, j--) {
        c = s[i];
        s[i] = s[j];
        s[j] = c;
    }
}
\end{minted}

\textbf{itoa.h}
\begin{minted}{c}
 void itoa_fcn(int n, char s[]);
\end{minted}




\clearpage
\section{Cheating Sheet - Toggle LEDs}
\label{appendix:E}
\begin{minted}{python}
# LEDs toggle every 1 second

import time


def ledRGBon():
        red = open("/sys/class/leds/led_r/brightness", "w")
        green = open("/sys/class/leds/led_g/brightness", "w")
        blue = open("/sys/class/leds/led_b/brightness", "w")

        red.write(str(1))
        green.write(str(1))
        blue.write(str(1))

        red.close()
        green.close()
        blue.close()


def ledRGBoff():
        red = open("/sys/class/leds/led_r/brightness", "w")
        green = open("/sys/class/leds/led_g/brightness", "w")
        blue = open("/sys/class/leds/led_b/brightness", "w")

        red.write(str(0))
        green.write(str(0))
        blue.write(str(0))

        red.close()
        green.close()
        blue.close()


while(1):
    ledRGBoff()
    time.sleep(1)
    ledRGBon()
    time.sleep(1)
\end{minted}




\clearpage
\section{Cheating Sheet - Read Accelerometer Data}
\label{appendix:D}
\input{appendixD}




\clearpage
\section{Cheating Sheet - Run the webserver}
\label{appendix:F}
\input{appendixF}





\end{appendices}


%-------------------------------------------------------------------------------
% REFERENCES
%-------------------------------------------------------------------------------
%\newpage
%\section*{Reading List}
%\addcontentsline{toc}{section}{Reading List}




  

\end{document}

%-------------------------------------------------------------------------------
% SNIPPETS
%-------------------------------------------------------------------------------

%\begin{figure}[!ht]
%	\centering
%	\includegraphics[width=0.8\textwidth]{file_name}
%	\caption{}
%	\centering
%	\label{label:file_name}
%\end{figure}

%\begin{figure}[!ht]
%	\centering
%	\includegraphics[width=0.8\textwidth]{graph}
%	\caption{Blood pressure ranges and associated level of hypertension (American Heart Association, 2013).}
%	\centering
%	\label{label:graph}
%\end{figure}

%\begin{wrapfigure}{r}{0.30\textwidth}
%	\vspace{-40pt}
%	\begin{center}
%		\includegraphics[width=0.29\textwidth]{file_name}
%	\end{center}
%	\vspace{-20pt}
%	\caption{}
%	\label{label:file_name}
%\end{wrapfigure}

%\begin{wrapfigure}{r}{0.45\textwidth}
%	\begin{center}
%		\includegraphics[width=0.29\textwidth]{manometer}
%	\end{center}
%	\caption{Aneroid sphygmomanometer with stethoscope (Medicalexpo, 2012).}
%	\label{label:manometer}
%\end{wrapfigure}

%\begin{table}[!ht]\footnotesize
%	\centering
%	\begin{tabular}{cccccc}
%	\toprule
%	\multicolumn{2}{c} {Pearson's correlation test} & \multicolumn{4}{c} {Independent t-test} \\
%	\midrule	
%	\multicolumn{2}{c} {Gender} & \multicolumn{2}{c} {Activity level} & \multicolumn{2}{c} {Gender} \\
%	\midrule
%	Males & Females & 1st level & 6th level & Males & Females \\
%	\midrule
%	\multicolumn{2}{c} {BMI vs. SP} & \multicolumn{2}{c} {Systolic pressure} & \multicolumn{2}{c} {Systolic Pressure} \\
%	\multicolumn{2}{c} {BMI vs. DP} & \multicolumn{2}{c} {Diastolic pressure} & \multicolumn{2}{c} {Diastolic pressure} \\
%	\multicolumn{2}{c} {BMI vs. MAP} & \multicolumn{2}{c} {MAP} & \multicolumn{2}{c} {MAP} \\
%	\multicolumn{2}{c} {W:H ratio vs. SP} & \multicolumn{2}{c} {BMI} & \multicolumn{2}{c} {BMI} \\
%	\multicolumn{2}{c} {W:H ratio vs. DP} & \multicolumn{2}{c} {W:H ratio} & \multicolumn{2}{c} {W:H ratio} \\
%	\multicolumn{2}{c} {W:H ratio vs. MAP} & \multicolumn{2}{c} {\% Body fat} & \multicolumn{2}{c} {\% Body fat} \\
%	\multicolumn{2}{c} {} & \multicolumn{2}{c} {Height} & \multicolumn{2}{c} {Height} \\
%	\multicolumn{2}{c} {} & \multicolumn{2}{c} {Weight} & \multicolumn{2}{c} {Weight} \\
%	\multicolumn{2}{c} {} & \multicolumn{2}{c} {Heart rate} & \multicolumn{2}{c} {Heart rate} \\
%	\bottomrule
%	\end{tabular}
%	\caption{Parameters that were analysed and related statistical test performed for current study. BMI - body mass index; SP - systolic pressure; DP - diastolic pressure; MAP - mean arterial pressure; W:H ratio - waist to hip ratio.}
%	\label{label:tests}
%\end{table}